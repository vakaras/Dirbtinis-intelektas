\chapter{Tiesioginis išvedimas}

\section{Pseudo kodas}

\begin{description}
  \item[$R$] – nepanaudotų taisyklių aibė.
  \item[$P$] – prielaidų aibė.
  \item[$G$] – ieškomas tikslas.
\end{description}
\begin{program}
  \FUNCT |ieškok_rekursyviai|(R, P, G) \BODY
    \IF G \in P \THEN \EXIT(\true) \FI
    \COMMENT{Iš $R$ pašalinam visas taisykles, %
             kurių rezultatas jau yra $P$.}
    \FOREACH r \in R \; \DO
      \IF r.|prielaidos| \subseteq P
      \THEN \EXIT(|ieškok_rekursyviai|%
        (R \setminus \{r\}, P \cup \{r.|rezultatas|\}, G ))
      \FI
    \OD
    \EXIT(\false)%
  \ENDFUNCT
\end{program}

\section{Programos kodas}

\pythonai{source}{forwardchaining.ForwardChaining.solve}
\pythonai{source}{forwardchaining.ForwardChaining.recursion}

\section{Pavyzdžiai}

\subsection{Pirmas pavyzdys}

\begin{pythonaienv}[fc]
# Taisyklės:
FB Z                                    # R1: F, B → Z
CD F                                    # R2: C, D → F
A D                                     # R3: A → D
A L                                     # R4: A → L
L K                                     # R5: L → K
B A                                     # R6: B → A
D M                                     # R7: D → M
# Faktai:
ABC
# Tikslas:
Z
\end{pythonaienv}

\subsection{Antras pavyzdys}

\begin{pythonaienv}[fc]
# Taisyklės:
G Z                                     # R1: G → Z
A G                                     # R2: A → G
A B                                     # R3: A → B
B C                                     # R4: B → C
C D                                     # R5: C → D
D Z                                     # R6: D → Z
# Faktai:
A
# Tikslas:
Z
\end{pythonaienv}

\subsection{Trečias pavyzdys}

\begin{pythonaienv}[fc]
# Taisyklės:
CD F                                    # R1: C, D → F
A D                                     # R2: A → D
A L                                     # R3: A → L
L K                                     # R4: L → K
B A                                     # R5: B → A
D M                                     # R6: D → M
FB Z                                    # R7: F, B → Z
# Faktai:
ABC
# Tikslas:
Z
\end{pythonaienv}

\subsection{Ketvirtas pavyzdys}

\begin{pythonaienv}[fc]
# Taisyklės:
CD F                                    # R1: C, D → F
A D                                     # R2: A → D
A L                                     # R3: A → L
L K                                     # R4: L → K
B A                                     # R5: B → A
D M                                     # R6: D → M
FB Z                                    # R7: F, B → Z
# Faktai:
ABC
# Tikslas:
H
\end{pythonaienv}

\subsection{Penktas pavyzdys}

\begin{pythonaienv}[fc]
# Taisyklės:
CD F                                    # R1: C, D → F
A D                                     # R2: A → D
A L                                     # R3: A → L
L K                                     # R4: L → K
B A                                     # R5: B → A
D M                                     # R6: D → M
FB Z                                    # R7: F, B → Z
# Faktai:
ABC
# Tikslas:
B
\end{pythonaienv}

\subsection{Šeštas pavyzdys}

\begin{pythonaienv}[fc]

# Taisyklės:
A L                                     # R1: A → L
L K                                     # R2: L → K
D A                                     # R3: D → A
D M                                     # R4: D → M
EFB Z                                   # R5: E, F, B → Z
ECD F                                   # R6: E, C, D → F
EA D                                    # R7: E, A → D

# Faktai:
A B C E

# Tikslas:
Z

\end{pythonaienv}
