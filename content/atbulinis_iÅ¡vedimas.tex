\chapter{Atbulinis išvedimas}

Pradedam nuo galo. Į Backtrace išvedame viską. Į atsakymą išvedame tik tai
ko reikia.

\section{Pseudo kodas}

\label{sec:bc:pseudo}

Pradiniai duomenys:
\begin{description}
  \item[$R$] – taisyklių aibė;
  \item[$F$] – pradinių faktų aibė;
  \item[$G$] – ieškomas tikslas.
\end{description}

Rezultatas:
\begin{description}
  \item[$Q$] – panaudotų taisyklių seka.
\end{description}

Kiti kintamieji:
\begin{description}
  \item[$P$] – ieškomų tikslų aibė.
\end{description}

\begin{algorithmic}[1]
  \Function{atbulinis išvedimas}{$R, F, G, P := \left\{ G \right\}$}
    \If{$G \in F$}
      \State \Return $\emptyset$
    \Else
      \ForAll{$r \in R \land r$ išvada yra $G \land$ nė viena iš $r$
              prielaidų $\not\in P$}
        \State $Q := \left( \right)$
        \ForAll{$g \in r$ prielaidos}
          \State $K := $ atbulinis išvedimas$(%
            R\setminus \left\{ r \right\}, F, g,%
            P \cup \left\{ g \right\})$;
          \If{$K =$ nesėkmė}
            \State $Q :=$ nesėkmė;
            \State break
          \Else
            \State $Q := Q \cup K$;
          \EndIf
          \If{$Q \neq$ nesėkmė}
            \State \Return $Q$;
          \EndIf
        \EndFor
      \EndFor
      \State \Return nesėkmė;
    \EndIf
  \EndFunction
\end{algorithmic}
