\chapter{Tiesioginis išvedimas}

\section{Pseudo kodas}

\begin{program}
  \rcomment{$R$ – nepanaudotų taisyklių aibė.}
  \rcomment{$F$ – faktų (pradinių ir išvestų) aibė.}
  \rcomment{$G$ – ieškomas tikslas.}
  \FUNCT |ieškok_rekursyviai|(R, F, G) \BODY
    \IF G \in F \THEN \EXIT(\true) \FI
    \COMMENT{Iš $R$ pašalinam visas taisykles, %
             kurių rezultatas jau yra $F$.}
    \FOREACH r \in R \DO
      \IF r.|prielaidos| \subseteq F
      \THEN \EXIT(|ieškok_rekursyviai|%
        (R \setminus \{r\}, F \cup \{r.|rezultatas|\}, G ))
      \FI
    \OD
    \EXIT(\false)%
  \ENDFUNCT
\end{program}

\section{Programos kodas}

\pythonai{source}{forwardchaining.ForwardChaining.solve}
\pythonai{source}{forwardchaining.ForwardChaining.recursion}

\section{Pavyzdžiai}

\subsection{Pirmas pavyzdys}

\begin{pythonaienv}[fc]
# Taisyklės:
FB Z
CD F
A D
A L
L K
B A
D M
# Faktai:
ABC
# Tikslas:
Z
\end{pythonaienv}

\subsection{Antras pavyzdys}

\begin{pythonaienv}[fc]
# Taisyklės:
G Z
A G
A B
B C
C D
D Z
# Faktai:
A
# Tikslas:
Z
\end{pythonaienv}

\subsection{Trečias pavyzdys}

\begin{pythonaienv}[fc]
# Taisyklės:
CD F
A D
A L
L K
B A
D M
FB Z
# Faktai:
ABC
# Tikslas:
Z
\end{pythonaienv}
