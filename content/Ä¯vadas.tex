\chapter{Įvadas}

Tiesioginis ir atbulinis išvedimai yra vieni iš pagrindinių dirbtinio
intelekto samprotavimo būdų produkcijų sistemoje.

Tiesioginio ir atbulinio išvedimų uždavinys yra, turint trejetą
$<R, F, G>$, kur
\begin{description}
  \item[$R$] – taisyklių aibė;
  \item[$F$] – pradinių faktų aibė;
  \item[$G$] – ieškomas tikslas (faktas, kurį norime įrodyti),
\end{description}
rasti tokią taisyklių (kurios būtų iš aibės $R$) seką be pasikartojimų,
kurią pritaikius pradinių faktų aibei gautume faktų aibę, kuriai
priklauso mūsų ieškomas tikslas $G$. Taisyklių aibei bendruoju
atveju priklauso $n$ taisyklių, pavidalo: $Ri: A \to B$, kur $A$ yra
faktų aibė, o $B$ – pagal taisyklę $Ri$ iš faktų $A$ išvedamas
faktas.

Tiesioginio išvedimo sistema bando išvesti ieškomąjį faktą $G$
tikrindama visas taisykles ir iš jų pritaikydama tas, kurių
prielaidos yra tarp jau išvestų faktų ir kurias panaudojant išvedamų
faktų dar nėra tarp rezultatų, tol kol yra gaunamas ieškomas tikslas
arba nebėra taisyklių, kurias galima būtų pritaikyti.

Atbulinio išvedimo sistema veikia iš priešingos pusės: pradeda
nuo tikslo, ieško taisyklės, kurios išvada yra tikslas, o tada
rekursyviai bando išvesti kiekvieną trūkstamą tos taisyklės
prielaidą.

\section{Realizacijos aprašymas}

Sistema parašyta naudojant Python programavimo kalbą. Sistema susideda
iš šių modulių:
\begin{enumerate}
  \item \verb|utils| – pagalbinės funkcijos ir klasės;
  \item \verb|forwardchaining| – modulis, kuriame realizuotas
    tiesioginio išvedimo funkcionalumas;
  \item \verb|backwardchaining| – modulis, kuriame realizuotas
    atbulinio išvedimo funkcionalumas;
  \item \verb|main| – vykdomasis modulis.
\end{enumerate}

\begin{sloppypar}
Visi duomenys (produkcijų sistema) yra saugomi klasėje
\verb|ProductionSystem|. Klasių \verb|ForwardChaining| ir
\verb|BackwardChaining| tipo objektai gauna rezultatą (taisykles,
kurias reikia pritaikyti norint pasiekti tikslą) manipuliuodami
susikurtu \verb|ProductionSystem| objektu. Visus duomenis sistema
išveda \LaTeX transliatoriui suprantamu formatu: dėl to visi šiame
dokumente pateikti pavyzdžiai iš tiesų yra sistemos išvestis.
Konkrečius programai pateiktus duomenis galima pamatyti atsidarius šio
dokumento išeities tekstą. (Buvo nuspręsta jų dokumente nerodyti,
nes sistema darbo pradžioje išveda gautus duomenis – tai yra,
informacija būtų dubliuojama.)
\end{sloppypar}

\section{Klasių išskirstymas į modulius}

\pythonai{structure}{}

\subsection{Klasė „Rule“}

\begin{minted}[linenos,texcl]{python}
class Rule:
    """ Išvedimo taisyklė.
    """

    def __init__(self, result, premises, index=None):
        self.result = result
        self.premises = list(premises)
        self.index = index
\end{minted}

\subsection{Klasė „ProductionSystem“}

\begin{minted}[linenos,texcl]{python}
class ProductionSystem:
    """ Produkcijų sistema. (Trejetas <R, F, G>.)
    """

    def __init__(self, rules, facts, goal):
        self.rules = rules
        self.facts = facts
        self.goal = goal
\end{minted}
